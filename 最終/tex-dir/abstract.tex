近年,Webアプリケーションの開発現場では,
短期間での実装や頻繁な機能追加が求められる一方で,
十分なセキュリティ対策が施されないまま運用される事例が増加している.
特に,小規模なWebアプリケーションや学習用途のシステムにおいては,
入力値検証やセッション管理の不備といった基本的な脆弱性が見過ごされやすく,
早期に検出・修正を支援する技術の重要性が高まっている.
このような背景のもと,
近年は大規模言語モデル(Large Language Models: LLM)を用いた
ソースコード解析や脆弱性検出への応用が注目されている.
しかし,LLMによる脆弱性検出性能は,モデルの学習状態や知識付与の方法,
外部情報の利用有無によって大きく左右されることが指摘されており,
それらの違いが検出精度や誤検知傾向に与える影響については,十分な比較・評価が行われていない.

本研究では,Webアプリケーションに対する脆弱性検出手法の高度化を目的として,
PHPプログラムを対象に,素のLLM,ファインチューニングを施したLLM,
およびRetrieval-Augmented Generation(RAG)構成LLMの三手法を比較評価する.
具体的には,XSS,CSRF,セッション管理の不備といった代表的なWeb脆弱性を対象とし,
Precision,Recall,F1-score,ROC曲線およびAUCを用いて検出性能を定量的に評価するとともに,
誤検知の発生傾向や提示される修正案の特徴について分析を行う.

本研究により,LLMの構成や知識付与方法の違いが,
脆弱性検出性能および実用上の有効性に与える影響を明らかにし,
LLMを用いた脆弱性検出手法の有効性と限界を整理する.これらの知見は,
今後のWebアプリケーション開発におけるセキュリティ支援技術の設計指針に資するものと期待される.
